this note will show how to increase the sled gradient by varying q e the external q of the sled cavity by increasing its q 0 and by increasing the compression ratio if varying the external q is to be effective then the copper losses should be small so that q 0 q e methods of varying q e will be indicated but no experimental data will be presented if we increase the klystron pulse width from 3.5 to 5  s and increase q 0 from the present 100000 to 300000 then the gradient increases by 19 and the beam energy increases from 50 to 60 gev this note will also discuss sled operation at 11424 mhz the nlc frequency without q e switching using sled at 11424 mhz increases the slac gradient from 21 mv m to 34 mv m and at the same repetition rate uses about 1 5 of rf average power if we also double the compression ratio we reach 47 mv m and over 100 gev beam energy latex original ap138 tex documentclass 12pt article this note will show how to increase the sled gradient by varying q_ e the external q of the sled cavity by increasing its q_ 0 and by increasing the compression ratio if varying the external q is to be effective then the copper losses should be small so that q_ 0 q_ e methods of varying q_ e will be indicated but no experimental data will be presented if we increase the klystron pulse width from 3.5 to 5 mu s and increase q_ 0 from the present 100000 to 300000 then the gradient increases by 19 and the beam energy increases from 50 to 60 gev this note will also discuss sled operation at 11424 mhz the nlc frequency without q_ e switching using sled at 11424 mhz increases the slac gradient from 21 mv m to 34 mv m and at the same repetition rate uses about 1 5 of rf average power if we also double the compression ratio we reach 47 mv m and over 100 gev beam energy end document
